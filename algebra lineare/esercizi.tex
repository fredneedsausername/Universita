
\documentclass{article}

\usepackage{amsmath}
\usepackage[italian]{babel}
\usepackage[a4paper,top=2cm,bottom=2cm,left=3cm,right=3cm,marginparwidth=1.75cm]{geometry}
\usepackage[colorlinks=true, allcolors=blue]{hyperref}
\usepackage{graphicx}
\usepackage{amssymb}
\usepackage{xcolor}

\usepackage{tcolorbox}
\tcbuselibrary{theorems}

% Blocco 'Definizione' come nel file originale
\newtcbtheorem{definition}{Definizione}
{
    colback=magenta!10!white,
    colframe=magenta!80!black,
    fonttitle=\bfseries
}
{def}

% Blocco 'Teorema' come nel file originale
\newtcbtheorem{theorem}{Teorema}
{
    colback=blue!10!white,
    colframe=blue!80!black,
    fonttitle=\bfseries
}
{thm}

% NUOVO: Blocco 'Esercizio' per le domande
\newtcbtheorem{exercise}{Esercizio}
{
    colback=green!5!white,
    colframe=green!60!black,
    fonttitle=\bfseries
}
{ex}

\title{Esercizi di Algebra Lineare}
\author{La tua guida per l'auto-valutazione}

\begin{document}
\maketitle

{
    \hypersetup{linkcolor=black}
    \tableofcontents
}

\section{Vettori e Spazi Vettoriali}

\begin{exercise}{Vero o Falso?}{ex:vero_falso_spazi}
    Valuta le seguenti affermazioni, giustificando la tua risposta.
    \begin{enumerate}
        \item L'insieme dei vettori di $\mathbb{R}^2$ della forma $\begin{pmatrix} x \\ y \end{pmatrix}$ con $x \ge 0$ e $y \ge 0$ (vettori del primo quadrante) è uno spazio vettoriale.
        \item L'insieme delle matrici $2 \times 2$ invertibili è uno spazio vettoriale.
        \item L'insieme $\mathbb{Q}$ (numeri razionali) con le usuali operazioni di somma e prodotto è uno spazio vettoriale sul campo $\mathbb{R}$ (numeri reali).
        \item L'insieme dei polinomi di grado \textbf{esattamente} 2 non è uno spazio vettoriale.
    \end{enumerate}
\end{exercise}

\begin{exercise}{Verifica delle Proprietà}{ex:proprieta_spazio}
    Considera l'insieme $V = \mathbb{R}^2$ con le seguenti operazioni "anomale":
    \begin{itemize}
        \item \textbf{Addizione:} $\begin{pmatrix} x_1 \\ y_1 \end{pmatrix} \oplus \begin{pmatrix} x_2 \\ y_2 \end{pmatrix} = \begin{pmatrix} x_1+x_2 \\ y_1+y_2 \end{pmatrix}$ (la solita addizione).
        \item \textbf{Moltiplicazione per scalare:} $\lambda \odot \begin{pmatrix} x \\ y \end{pmatrix} = \begin{pmatrix} \lambda x \\ y \end{pmatrix}$.
    \end{itemize}
    Questo è uno spazio vettoriale sul campo $\mathbb{R}$? Se no, quali proprietà non sono soddisfatte?
\end{exercise}

\section{Sottospazi Vettoriali}

\begin{exercise}{Identificazione di Sottospazi}{ex:identifica_sottospazi}
    Determina quali dei seguenti sottoinsiemi $W$ sono sottospazi vettoriali dello spazio $V$ indicato.
    \begin{enumerate}
        \item $V = \mathbb{R}^3$, $W = \{ (x, y, z) \in \mathbb{R}^3 \mid x + y - 2z = 0 \}$.
        \item $V = \mathbb{R}^3$, $W = \{ (x, y, z) \in \mathbb{R}^3 \mid z = 1 \}$.
        \item $V = M(2, 2, \mathbb{R})$, $W = \{ A \in M(2, 2, \mathbb{R}) \mid \text{det}(A) = 0 \}$.
        \item $V = \mathbb{R}[x]_{\le 3}$, $W = \{ p(x) \in V \mid p(1) = 0 \}$.
    \end{enumerate}
\end{exercise}

\begin{exercise}{Unione e Intersezione}{ex:unione_intersezione}
    Dimostra che l'intersezione di due sottospazi vettoriali di $V$ è sempre un sottospazio vettoriale. L'unione di due sottospazi è anch'essa sempre un sottospazio? Se no, fornisci un controesempio.
\end{exercise}

\section{Combinazioni Lineari, Basi e Dimensione}

\begin{exercise}{Verifica di Dipendenza Lineare}{ex:dipendenza_lineare}
    Stablisci se il seguente insieme di vettori in $\mathbb{R}^3$ è linearmente indipendente. Se non lo è, esprimi uno dei vettori come combinazione lineare degli altri.
    \[ S = \left\{ v_1 = \begin{pmatrix} 1 \\ 2 \\ 1 \end{pmatrix}, v_2 = \begin{pmatrix} 0 \\ 1 \\ -1 \end{pmatrix}, v_3 = \begin{pmatrix} 2 \\ 3 \\ 3 \end{pmatrix} \right\} \]
\end{exercise}

\begin{exercise}{Estrarre una Base}{ex:estrarre_base}
    Utilizzando l'algoritmo di eliminazione di Gauss per colonne, estrai una base per il sottospazio $W = \text{SPAN}(v_1, v_2, v_3, v_4)$, dove:
    \[ v_1 = \begin{pmatrix} 1 \\ 0 \\ -1 \end{pmatrix}, \quad v_2 = \begin{pmatrix} 2 \\ 1 \\ 0 \end{pmatrix}, \quad v_3 = \begin{pmatrix} 0 \\ 1 \\ 2 \end{pmatrix}, \quad v_4 = \begin{pmatrix} 1 \\ 1 \\ 1 \end{pmatrix} \]
    Qual è la dimensione di $W$?
\end{exercise}

\begin{exercise}{Completamento a Base}{ex:completamento_base}
    L'insieme di vettori $S = \left\{ \begin{pmatrix} 1 \\ 1 \\ 0 \end{pmatrix}, \begin{pmatrix} 0 \\ 1 \\ 1 \end{pmatrix} \right\}$ è linearmente indipendente in $\mathbb{R}^3$? Se sì, trova un terzo vettore da aggiungere a $S$ per formare una base di $\mathbb{R}^3$.
\end{exercise}

\begin{exercise}{Basi in Spazi di Polinomi}{ex:base_polinomi}
    Verifica che l'insieme $\{1, x-1, (x-1)^2\}$ è una base per lo spazio vettoriale $\mathbb{R}[x]_{\le 2}$. Successivamente, scrivi il polinomio $p(x) = 2x^2 + 3x - 1$ come combinazione lineare dei vettori di questa base.
\end{exercise}

\section{Matrici}

\begin{exercise}{Proprietà delle Matrici}{ex:proprieta_matrici}
    Rispondi alle seguenti domande.
    \begin{enumerate}
        \item Se $A$ è una matrice antisimmetrica, quanto vale la sua traccia?
        \item Costruisci un esempio di matrice $3 \times 3$ che sia contemporaneamente triangolare superiore e simmetrica. Che tipo di matrice ottieni?
        \item Dimostra che lo spazio delle matrici simmetriche $n \times n$ è un sottospazio vettoriale di $M(n, n, \mathbb{K})$.
    \end{enumerate}
\end{exercise}

\begin{exercise}{Decomposizione Simmetrica-Antisimmetrica}{ex:decomposizione}
    Ogni matrice quadrata $A$ può essere scritta come la somma di una matrice simmetrica $S$ e una matrice antisimmetrica $T$.
    \begin{enumerate}
        \item Trova le formule generali per $S$ e $T$ in termini di $A$ e della sua trasposta $A^T$. (Suggerimento: considera le combinazioni $A+A^T$ e $A-A^T$).
        \item Applica queste formule per decomporre la seguente matrice:
        \[ A = \begin{pmatrix} 1 & 5 & 3 \\ 2 & 4 & 0 \\ -1 & 6 & 7 \end{pmatrix} \]
    \end{enumerate}
\end{exercise}

\section{Isomorfismi}

\begin{exercise}{Verifica di Isomorfismo}{ex:isomorfismi}
    Quali dei seguenti spazi vettoriali sono isomorfi tra loro? Giustifica la risposta senza costruire l'isomorfismo esplicito.
    \begin{itemize}
        \item $V_1 = \mathbb{R}^4$
        \item $V_2 = \mathbb{R}[x]_{\le 4}$
        \item $V_3 = M(2, 2, \mathbb{R})$
        \item $V_4 = \{ A \in M(2, 2, \mathbb{R}) \mid A \text{ è simmetrica} \}$
    \end{itemize}
\end{exercise}

\begin{exercise}{Costruzione di un Isomorfismo}{ex:costruzione_isomorfismo}
    Considera lo spazio $W$ delle matrici $2 \times 2$ triangolari superiori.
    \begin{enumerate}
        \item Trova una base per $W$ e determina la sua dimensione.
        \item In base al risultato, a quale spazio $\mathbb{R}^n$ è isomorfo?
        \item Scrivi esplicitamente la funzione di isomorfismo $f: W \to \mathbb{R}^n$ che associa a ogni matrice in $W$ il vettore delle sue coordinate rispetto alla base che hai scelto.
    \end{enumerate}
\end{exercise}

\end{document}