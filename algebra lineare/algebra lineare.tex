\documentclass{article}

\usepackage{amsmath}
\usepackage[italian]{babel}
\usepackage[a4paper,top=2cm,bottom=2cm,left=3cm,right=3cm,marginparwidth=1.75cm]{geometry}
\usepackage[colorlinks=true, allcolors=blue]{hyperref}
\usepackage{graphicx}
\usepackage{amssymb}
\usepackage{xcolor}

\usepackage{tcolorbox}
\tcbuselibrary{theorems}

% This creates a new tcolorbox for definitions
\newtcbtheorem{definition}{Definizione} % {box name}{display name}
{
    colback=magenta!10!white,
    colframe=magenta!80!black,
    fonttitle=\bfseries
}
{def} % This is a prefix for the label, so you can reference it like \ref{def:vector_space}

\title{Algebra Lineare}
\author{Federico Veronesi}

\begin{document}
\maketitle

{
    \hypersetup{linkcolor=black}
    \tableofcontents
}

\section{Vettori}

\begin{definition}{Vettore}{vettore}
    Un vettore di campo $\mathbb{K}$ e dimensione $n$ è un insieme di elementi $a_1, a_2, \dots , a_n$, ciascuno di campo $\mathbb{K}$. In notazione: $\begin{pmatrix} x1 \\ \vdots \\ x_n \end{pmatrix}$ come ad esempio $\begin{pmatrix} 2 \\ -3.5 \\ \pi \end{pmatrix}$ di dimensione $3$ e campo $\mathbb{R}$
\end{definition}

\begin{definition}{Spazio Vettoriale}{spazio_vettoriale}
    Uno \textbf{spazio vettoriale} su un campo $\mathbb{K}$ è un insieme di \hyperlink{def:vettore}{vettori} $V$ che definisce due operazioni:
    \begin{enumerate}
        \item Addizione fra vettori: $V + V \to V$
        \item Moltiplicazione scalare: $\mathbb{K} \times V \to V$
    \end{enumerate}
\end{definition}

Alcuni esempi di spazi vettoriali sono i seguenti:
\begin{itemize}
    \item $\mathbb{R} [x]$ L'insieme dei polinomi
    \item $\mathbb{R} [x]^{\le n}$ L'insieme dei polinomi di grado $\le n$
    \item $C(\mathbb{R}, \mathbb{R})$ Le funzioni continue da $\mathbb{R}$ in $\mathbb{R}$
    \item $M(m, n, \mathbb{K})$ L'insieme delle matrici $m \times n$ a coefficiente di campo $\mathbb{K}$
\end{itemize}

\subsection{Sottospazi Vettoriali}

\begin{definition}{Sottospazio Vettoriale}{sottospazio_vettoriale}
    Un sottospazio vettoriale è uno \hyperlink{def:spazio_vettoriale}{spazio vettoriale} $W$ incluso in uno spazio vettoriale $V$. Ogni operazione fra vettori di $W$ ritorna un vettore incluso in $W$. Ogni sottospazio vettoriale include l'elemento neutro dello spazio vettoriale.
\end{definition}

\subsection{Combinazioni Lineari}

\section{Matrici}

\begin{definition}{Matice}{matrice}
    Una matrice è un \hyperlink{def:vettore}{vettore} bidimensionale $m \times n$
\end{definition}

\subsection{Tipologie di Matrici}

\end{document}