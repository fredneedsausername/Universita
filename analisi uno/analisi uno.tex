\documentclass{article}

\usepackage{amsmath}
\usepackage[italian]{babel}
\usepackage[a4paper,top=2cm,bottom=2cm,left=3cm,right=3cm,marginparwidth=1.75cm]{geometry}
\usepackage[colorlinks=true, allcolors=blue]{hyperref}
\usepackage{graphicx}
\usepackage{amssymb}
\usepackage{xcolor}

\usepackage{tcolorbox}
\tcbuselibrary{theorems}

% This creates a new tcolorbox for definitions
\newtcbtheorem{definition}{Definizione} % {box name}{display name}
{
    colback=magenta!10!white,
    colframe=magenta!80!black,
    fonttitle=\bfseries
}
{def} % This is a prefix for the label, so you can reference it like \ref{def:vector_space}

\newtcbtheorem{theorem}{Teorema} % {box name}{display name}
{
    colback=blue!10!white,
    colframe=blue!80!black,
    fonttitle=\bfseries
}
{thm}

\newtcbtheorem{exercise}{Esercizio} % {box name}{display name}
{
    colback=green!10!white,
    colframe=green!80!black,
    fonttitle=\bfseries
}
{exc}

\title{Algebra Lineare}
\author{Federico Veronesi}

\begin{document}
\maketitle

{
    \hypersetup{linkcolor=black}
    \tableofcontents
}

\section{Logaritmi}

\subsection{Definizione}

\begin{definition}{Logaritmo}{def:logaritmo}
    Dati due numeri reali positivi $a$ e $b$, con $b \neq 1$, il \textbf{logaritmo} in base $b$ di $a$ è l'esponente $c$ a cui elevare $b$ per ottenere $a$.
    \\[1em]
    In notazione:
    $$ \log_b(a) = c \iff b^c = a $$
    Dove:
    \begin{itemize}
        \item $b$ è la \textbf{base} ($b > 0, b \neq 1$)
        \item $a$ è l' \textbf{argomento} ($a > 0$)
        \item $c$ è il \textbf{logaritmo}
    \end{itemize}
    \vspace{1em}
    "Il logaritmo è l'operazione inversa dell'elevamento a potenza. Risponde semplicemente alla domanda: 'a quale numero devo elevare la base per ottenere l'argomento?'"
    \\[1em]
    "Ad esempio, $\log_2(8) = 3$ perché $2^3 = 8$."
\end{definition}

\subsection{Proprietà}

\begin{theorem}{Proprietà dei Logaritmi}{thm:log_properties}
    Siano $x, y > 0$, $b > 0$ e $b \neq 1$, e $k \in \mathbb{R}$. Valgono le seguenti proprietà:
    \begin{itemize}
        \item \textbf{Logaritmo di un prodotto}: $\log_b(x \cdot y) = \log_b(x) + \log_b(y)$
        \item \textbf{Logaritmo di un quoziente}: $\log_b\left(\frac{x}{y}\right) = \log_b(x) - \log_b(y)$
        \item \textbf{Logaritmo di una potenza}: $\log_b(x^k) = k \cdot \log_b(x)$
        \item \textbf{Formula del cambio di base}: $\log_a(x) = \frac{\log_b(x)}{\log_b(a)}$
    \end{itemize}
    Inoltre, valgono questi casi notevoli:
    \begin{itemize}
        \item $\log_b(1) = 0$ (perché $b^0 = 1$)
        \item $\log_b(b) = 1$ (perché $b^1 = b$)
    \end{itemize}
    \vspace{1em}
    "Le proprietà dei logaritmi ci permettono di 'semplificare' le operazioni: trasformano le moltiplicazioni in somme, le divisioni in sottrazioni e le potenze in moltiplicazioni."
\end{theorem}


\subsection{Esercizi}

\begin{exercise}{Proprietà}{}
    Enuncia tutte le proprietà dei logaritmi.
\end{exercise}

\section{Funzioni}

\subsection{Definizioni}

\begin{definition}{Funzione Iniettiva}{funzione_iniettiva}
    Una funzione è iniettiva quando tutti gli elementi del dominio hanno immagine diversa.

    In notazione: $ \forall a_1, a_2 \in \mathbb{R}: a_1 \neq a_2 \implies f(a_1) \neq f(a_2)$
\end{definition}

\begin{definition}{Funzione Suriettiva}{funzione_suriettiva}
    Una funzione è suriettiva quando non esistono elementi \textit{superflui} nel codominio.

    In notazione: $ \forall b \in B \, \exists \, a \in A: f(a) = b $
\end{definition}

\begin{definition}{Funzione Bigettiva}{funzione_bigettiva}
    Una funzione è bigettiva quando è sia \hyperref[def:funzione_iniettiva]{iniettiva} che \hyperref[def:funzione_suriettiva]{suriettiva}
\end{definition}

\begin{definition}{Funzione Inversa}{funzione_inversa}
    Una funzione $g$ è l'inversa di $f$ quando $g(f(a)) = a$
\end{definition}

\section{Limiti}

\subsection{Preambolo}

\begin{definition}{Massimo e Minimo di Un Insieme}{max_min}
    Dato un insieme $A \subseteq \mathbb{R}$:
    \begin{itemize}
        \item Un numero $M$ è il \textbf{massimo} di $A$ se $M \in A$ e $\forall a \in A, a \le M$.
        \item Un numero $m$ è il \textbf{minimo} di $A$ se $m \in A$ e $\forall a \in A, a \ge m$.
    \end{itemize}
    "$max((0,1))$ non esiste, perché non c'è un elemento nell'insieme che sia più grande di tutti gli altri (1 è escluso). Invece, $max([0,1]) = 1$."
\end{definition}

\begin{definition}{Maggiorante e Minorante}{maggiorante_minorante}
    Dato un insieme $A \subseteq \mathbb{R}$:
    \begin{itemize}
        \item Un numero $K \in \mathbb{R}$ è un \textbf{maggiorante} di $A$ se $\forall a \in A, a \le K$.
        \item Un numero $k \in \mathbb{R}$ è un \textbf{minorante} di $A$ se $\forall a \in A, a \ge k$.
    \end{itemize}
    "L'insieme dei maggioranti di $(0,1)$ è $[1, +\infty)$. L'insieme dei minoranti è $(-\infty, 0]$"
\end{definition}

\begin{definition}{Estremi Superiore e Inferiore}{sup_inf}
    Dato un insieme $A \subseteq \mathbb{R}$:
    \begin{itemize}
        \item L'\textbf{estremo superiore} di $A$, denotato con $\sup(A)$, è il più piccolo dei maggioranti.
        \item L'\textbf{estremo inferiore} di $A$, denotato con $\inf(A)$, è il più grande dei minoranti.
    \end{itemize}
    "$sup((0,1)) = 1$ e $\inf((0,1)) = 0$"
\end{definition}


\subsection{Definizione}

\begin{definition}{Limite di Successione}{limite}
    Nota: esistono limiti convergenti e divergenti. Il professore ha solo introdotto quella del limite convergente, che segue adesso con spiegazione.
    \\[1em]
    $\forall \epsilon > 0, \exists N \in \mathbb{N} : \forall n > N, |a_n - L| < \epsilon$
    \\[1em]
    "Giochiamo a dire il numero più piccolo chiamato epsilon che puoi, perché ti garantisco che prima o poi io con questa successione mi avvicinerò talmente tanto a uno specifico valore chiamato L che la distanza fra lo specifico valore che cerco e il punto in cui la mia successione è arrivata sarà più piccola del tuo numero epsilon".
    \\[1em]
    "Con la mia successione mi avvicinerò tanto a uno specifico valore che qualunque distanza sarà maggiore della distanza da quello specifico valore".
\end{definition}

\subsection{Teoremi}

\begin{theorem}{Teorema Dell'Algebra dei Limiti}{}
    Siano $(a_n)$ e $(b_n)$ due successioni convergenti, con $\lim_{n \to \infty} a_n = L$ e $\lim_{n \to \infty} b_n = M$. Allora:
    \begin{itemize}
        \item $\lim_{n \to \infty} (a_n \pm b_n) = L \pm M$
        \item $\lim_{n \to \infty} (a_n \cdot b_n) = L \cdot M$
        \item $\lim_{n \to \infty} (k \cdot a_n) = k \cdot L$, per ogni costante $k \in \mathbb{R}$
        \item $\lim_{n \to \infty} \frac{a_n}{b_n} = \frac{L}{M}$, se $M \neq 0$ e $b_n \neq 0$ per ogni $n$.
    \end{itemize}

    "La operazioni fra limiti sono i limiti dei risultati delle operazioni"
\end{theorem}

\begin{theorem}{Teorema della Permanenza del Segno}{}
    Se una successione $(a_n)$ converge a un limite $L \neq 0$, allora esiste un indice $N \in \mathbb{N}$ tale che per ogni $n > N$, il termine $a_n$ ha lo stesso segno di $L$.

    "Nel processo di calcolare un limite, esiste un certo numero di passaggi della successione oltre il quale il segno della successione in quel numero di passaggi ha lo stesso segno del numero a cui converge il limite"

    "In un intorno di un numero diverso da zero c'è almeno un pezzettino con lo stesso segno di quel numero"
\end{theorem}

\begin{theorem}{Teorema del Confronto}{}
    Siano $(a_n)$, $(b_n)$, e $(c_n)$ tre successioni tali che $a_n \le b_n \le c_n$ definitivamente. Se $\lim_{n \to \infty} a_n = \lim_{n \to \infty} c_n = L$, allora anche $\lim_{n \to \infty} b_n = L$.

    "Immaginatelo con un grafico. Se due curve ne comprendono una in mezzo, quando queste due curve si toccano, allora si toccano anche con quella che sta in mezzo"
\end{theorem}

\subsection{Limiti Notevoli}

TODO (aspetto lezione che ce li mostrino)

\subsection{Esercizi}

\begin{exercise}{Prima Risoluzione di Limiti}{}
    Usando esclusivamente \hyperref[def:limite]{la definizione di limite di successione}, risolvi dei limiti.
\end{exercise}

\begin{exercise}{Definizioni}{}
    Spiega le definizioni in questo capitolo prima in modo rigoroso e poi intuitivo.
\end{exercise}

\end{document}