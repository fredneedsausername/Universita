\documentclass{article}

\usepackage{amsmath}
\usepackage[italian]{babel}
\usepackage[a4paper,top=2cm,bottom=2cm,left=3cm,right=3cm,marginparwidth=1.75cm]{geometry}
\usepackage[colorlinks=true, allcolors=blue]{hyperref}
\usepackage{graphicx}
\usepackage{amssymb}
\usepackage{xcolor}

\usepackage{tcolorbox}
\tcbuselibrary{theorems}

% Aggiungo pgfplots per disegnare i grafici
\usepackage{pgfplots}
\pgfplotsset{compat=1.18}

% Questo crea un nuovo tcolorbox per le definizioni
\newtcbtheorem{definition}{Definizione} % {nome box}{nome visualizzato}
{
    colback=magenta!10!white,
    colframe=magenta!80!black,
    fonttitle=\bfseries
}
{def} % Questo è un prefisso per la label

\newtcbtheorem{theorem}{Teorema} % {nome box}{nome visualizzato}
{
    colback=blue!10!white,
    colframe=blue!80!black,
    fonttitle=\bfseries
}
{thm}

% Box per le spiegazioni
\newtcbtheorem{explanation}{Spiegazione}{
    colback=green!10!white,
    colframe=green!60!black,
    fonttitle=\bfseries
}
{expl}

\title{Analisi Matematica}
\author{Federico Veronesi}

\begin{document}
\maketitle

{
    \hypersetup{linkcolor=black}
    \tableofcontents
}

\section{Logaritmi}

\subsection{Definizione}

\begin{definition}{Logaritmo}{logaritmo}
    Dati due numeri reali positivi $a$ e $b$, con $b \neq 1$, il \textbf{logaritmo} in base $b$ di $a$ è l'esponente $c$ a cui elevare $b$ per ottenere $a$. \\[1em]
    In notazione:
    $$ \log_b(a) = c \iff b^c = a $$
    Dove:
    \begin{itemize}
        \item $b$ è la \textbf{base} ($b > 0, b \neq 1$)
        \item $a$ è l' \textbf{argomento} ($a > 0$)
        \item $c$ è il \textbf{logaritmo}
    \end{itemize}
\end{definition}

Ad esempio, $\log_2(8) = 3$ poiché $2^3 = 8$.

\subsection{Proprietà}

\begin{theorem}{Proprietà dei Logaritmi}{log_properties}
    Siano $x, y > 0$, $b > 0$ e $b \neq 1$, e $k \in \mathbb{R}$. Valgono le seguenti proprietà:
    \begin{itemize}
        \item \textbf{Logaritmo di un prodotto}: $\log_b(x \cdot y) = \log_b(x) + \log_b(y)$
        \item \textbf{Logaritmo di un quoziente}: $\log_b\left(\frac{x}{y}\right) = \log_b(x) - \log_b(y)$
        \item \textbf{Logaritmo di una potenza}: $\log_b(x^k) = k \cdot \log_b(x)$
        \item \textbf{Formula del cambio di base}: $\log_a(x) = \frac{\log_b(x)}{\log_b(a)}$
    \end{itemize}
    Inoltre, valgono questi casi notevoli:
    \begin{itemize}
        \item $\log_b(1) = 0$
        \item $\log_b(b) = 1$
    \end{itemize}
\end{theorem}

\subsection{Grafici della Funzione Logaritmica}
Il grafico della funzione $f(x) = \log_b(x)$ dipende dalla base $b$.
\begin{itemize}
    \item \textbf{Se $b > 1$}, la funzione è crescente.
    \item \textbf{Se $0 < b < 1$}, la funzione è decrescente.
\end{itemize}
In entrambi i casi, il dominio è $(0, +\infty)$ e l'asse $y$ è un asintoto verticale.

\begin{figure}[!htbp]
\centering
\begin{tikzpicture}
\begin{axis}[
    axis lines=middle,
    xlabel=$x$,
    ylabel=$y$,
    title={Grafico di $y = \log_b(x)$ con $b>1$},
    domain=0.1:10,
    samples=100,
    legend pos=north west
]
\addplot[blue, thick] {ln(x)};
\legend{$\ln(x)$}
\end{axis}
\end{tikzpicture}
\caption{Esempio di funzione logaritmica con base $b=e > 1$.}
\end{figure}

\begin{figure}[!htbp]
\centering
\begin{tikzpicture}
\begin{axis}[
    axis lines=middle,
    xlabel=$x$,
    ylabel=$y$,
    title={Grafico di $y = \log_b(x)$ con $0<b<1$},
    domain=0.1:10,
    samples=100,
    legend pos=north east
]
\addplot[red, thick] {log2(x)/log2(0.5)};
\legend{$\log_{0.5}(x)$}
\end{axis}
\end{tikzpicture}
\caption{Esempio di funzione logaritmica con base $b=0.5 < 1$.}
\end{figure}


\section{Funzioni}

\subsection{Definizioni}

\begin{definition}{Funzione Iniettiva}{funzione_iniettiva}
    Una funzione si dice iniettiva se a elementi distinti del dominio corrispondono immagini distinte nel codominio. In notazione: $ \forall a_1, a_2 \in A: a_1 \neq a_2 \implies f(a_1) \neq f(a_2)$
\end{definition}

\begin{definition}{Funzione Suriettiva}{funzione_suriettiva}
    Una funzione si dice suriettiva quando ogni elemento del codominio è immagine di almeno un elemento del dominio. In notazione: $ \forall b \in B \, \exists \, a \in A: f(a) = b $
\end{definition}

\begin{definition}{Funzione Bigettiva}{funzione_bigettiva}
    Una funzione è bigettiva (o biunivoca) quando è sia iniettiva che suriettiva.
\end{definition}

\begin{definition}{Funzione Inversa}{funzione_inversa}
    Data una funzione bigettiva $f: A \to B$, la sua funzione inversa è la funzione $f^{-1}: B \to A$ tale che la composizione $f^{-1} \circ f$ è la funzione identità su $A$.
\end{definition}

\subsection{Funzione Esponenziale}
La funzione esponenziale è una funzione del tipo $f(x) = b^x$, dove la base $b$ è un numero reale positivo e diverso da 1. È la funzione inversa della funzione logaritmica.
Il grafico della funzione esponenziale dipende dalla base $b$:
\begin{itemize}
    \item \textbf{Se $b > 1$}, la funzione è crescente.
    \item \textbf{Se $0 < b < 1$}, la funzione è decrescente.
\end{itemize}
L'asse $x$ è un asintoto orizzontale per la funzione.

\begin{figure}[!htbp]
\centering
\begin{tikzpicture}
\begin{axis}[
    axis lines=middle,
    xlabel=$x$,
    ylabel=$y$,
    title={Grafico di $y = b^x$ con $b>1$},
    domain=-3:3,
    samples=100,
    legend pos=north west
]
\addplot[blue, thick] {exp(x)};
\legend{$e^x$}
\end{axis}
\end{tikzpicture}
\caption{Esempio di funzione esponenziale con base $b=e > 1$.}
\end{figure}

\begin{figure}[!htbp]
\centering
\begin{tikzpicture}
\begin{axis}[
    axis lines=middle,
    xlabel=$x$,
    ylabel=$y$,
    title={Grafico di $y = b^x$ con $0<b<1$},
    domain=-3:3,
    samples=100,
    legend pos=north east
]
\addplot[red, thick] {0.5^x};
\legend{$(0.5)^x$}
\end{axis}
\end{tikzpicture}
\caption{Esempio di funzione esponenziale con base $b=0.5 < 1$.}
\end{figure}


\section{Limiti}

\subsection{Preambolo}

\begin{definition}{Massimo e Minimo di Un Insieme}{max_min}
    Dato un insieme $A \subseteq \mathbb{R}$:
    \begin{itemize}
        \item Un numero $M$ è il \textbf{massimo} di $A$ se $M \in A$ e $\forall a \in A, a \le M$.
        \item Un numero $m$ è il \textbf{minimo} di $A$ se $m \in A$ e $\forall a \in A, a \ge m$.
    \end{itemize}
\end{definition}

Ad esempio, l'intervallo $(0,1)$ non ammette massimo, mentre l'intervallo $[0,1]$ ha massimo uguale a 1.

\begin{definition}{Maggiorante e Minorante}{maggiorante_minorante}
    Dato un insieme $A \subseteq \mathbb{R}$:
    \begin{itemize}
        \item Un numero $K \in \mathbb{R}$ è un \textbf{maggiorante} di $A$ se $\forall a \in A, a \le K$.
        \item Un numero $k \in \mathbb{R}$ è un \textbf{minorante} di $A$ se $\forall a \in A, a \ge k$.
    \end{itemize}
\end{definition}

\begin{definition}{Estremi Superiore e Inferiore}{sup_inf}
    Dato un insieme $A \subseteq \mathbb{R}$:
    \begin{itemize}
        \item L'\textbf{estremo superiore} di $A$, denotato con $\sup(A)$, è il più piccolo dei maggioranti.
        \item L'\textbf{estremo inferiore} di $A$, denotato con $\inf(A)$, è il più grande dei minoranti.
    \end{itemize}
\end{definition}

\subsection{Definizione di Limite}

\begin{definition}{Limite di Successione Convergente}{limite_convergente}
    Una successione $(a_n)$ converge al limite $L \in \mathbb{R}$ se:
    \\[1em]
    $\forall \epsilon > 0, \exists N \in \mathbb{N} : \forall n > N, |a_n - L| < \epsilon$
\end{definition}

\begin{explanation}{Cosa significa la definizione?}{}
    Questa definizione, per quanto astratta, esprime un'idea semplice. Dire che una successione tende a $L$ significa che i suoi termini si "avvicinano" sempre di più a $L$.
    La definizione lo formalizza così: possiamo rendere la distanza $|a_n - L|$ piccola a piacere (più piccola di un qualsiasi $\epsilon > 0$, anche piccolissimo), a patto di andare abbastanza avanti nella successione (scegliendo un indice $n$ più grande di un certo $N$). In altre parole, da un certo punto $N$ in poi, tutti i termini della successione si trovano in un piccolo intervallo di ampiezza $2\epsilon$ centrato su $L$.
\end{explanation}

\begin{definition}{Limite di Successione Divergente}{limite_divergente}
    Una successione $(a_n)$ si dice divergente se tende a $\pm\infty$.
    \begin{itemize}
        \item Diciamo che $(a_n)$ diverge a $+\infty$ se:
         $$ \forall M > 0, \exists N \in \mathbb{N} : \forall n > N, a_n > M $$
        \item Diciamo che $(a_n)$ diverge a $-\infty$ se:
         $$ \forall M < 0, \exists N \in \mathbb{N} : \forall n > N, a_n < M $$
    \end{itemize}
\end{definition}

\begin{explanation}{Cosa significa divergere?}{}
    Dire che una successione diverge a $+\infty$ significa che i suoi termini diventano arbitrariamente grandi. La definizione cattura questa idea: per quanto grande si scelga un numero $M$ (ad esempio $10^{100}$), è sempre possibile trovare un punto della successione (un indice $N$) dopo il quale tutti i termini $a_n$ sono ancora più grandi di $M$. Un discorso analogo vale per la divergenza a $-\infty$.
\end{explanation}

\subsection{Teoremi sui Limiti}

\begin{theorem}{Teorema Dell'Algebra dei Limiti}{}
    Siano $(a_n)$ e $(b_n)$ due successioni convergenti, con $\lim_{n \to \infty} a_n = L$ e $\lim_{n \to \infty} b_n = M$. Allora:
    \begin{itemize}
        \item $\lim_{n \to \infty} (a_n \pm b_n) = L \pm M$
        \item $\lim_{n \to \infty} (a_n \cdot b_n) = L \cdot M$
        \item $\lim_{n \to \infty} (k \cdot a_n) = k \cdot L$, per ogni costante $k \in \mathbb{R}$
        \item $\lim_{n \to \infty} \frac{a_n}{b_n} = \frac{L}{M}$, se $M \neq 0$ e $b_n \neq 0$ per ogni $n$.
    \end{itemize}
\end{theorem}

\begin{theorem}{Teorema della Permanenza del Segno}{}
    Se una successione $(a_n)$ converge a un limite $L \neq 0$, allora esiste un indice $N \in \mathbb{N}$ tale che per ogni $n > N$, il termine $a_n$ ha lo stesso segno di $L$.
\end{theorem}

\begin{theorem}{Teorema del Confronto}{}
    Siano $(a_n)$, $(b_n)$, e $(c_n)$ tre successioni tali che $a_n \le b_n \le c_n$ definitivamente. Se $\lim_{n \to \infty} a_n = \lim_{n \to \infty} c_n = L$, allora anche $\lim_{n \to \infty} b_n = L$.
\end{theorem}

\subsection{Forme di Indeterminazione}
L'algebra dei limiti non si può applicare direttamente quando le operazioni portano a una delle seguenti forme, dette \textbf{forme indeterminate} o \textbf{di indecisione}:
\begin{center}
    $ \left[ \frac{0}{0} \right], \quad \left[ \frac{\infty}{\infty} \right], \quad [0 \cdot \infty], \quad [+\infty - \infty], \quad [1^\infty], \quad [0^0], \quad [\infty^0] $
\end{center}
In questi casi, il limite deve essere calcolato con altri metodi, come l'uso dei limiti notevoli, il confronto tra infiniti o (per le funzioni) il teorema di De L'Hôpital.

\begin{theorem}{Limiti Notevoli}{limiti_notevoli}
    I seguenti limiti sono particolarmente utili per risolvere le forme indeterminate:
    \begin{itemize}
        \item $\lim_{x \to 0} \frac{\sin(x)}{x} = 1$
        \item $\lim_{x \to 0} \frac{1 - \cos(x)}{x^2} = \frac{1}{2}$
        \item $\lim_{x \to \pm\infty} \left(1 + \frac{1}{x}\right)^x = e$
        \item $\lim_{x \to 0} \frac{\ln(1+x)}{x} = 1$
        \item $\lim_{x \to 0} \frac{e^x - 1}{x} = 1$
        \item $\lim_{x \to 0} \frac{(1+x)^k - 1}{x} = k, \quad \forall k \in \mathbb{R}$
    \end{itemize}
\end{theorem}

\section{Esercizi Svolti}
In classe è stata risolta la seguente disequazione: $ \sqrt{|x-1|} \le 2-x $.

\end{document}