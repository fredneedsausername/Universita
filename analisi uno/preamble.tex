\usepackage{amsmath}
\usepackage[italian]{babel}
\usepackage[a4paper,top=2cm,bottom=2cm,left=3cm,right=3cm,marginparwidth=1.75cm]{geometry}
\usepackage[colorlinks=true, allcolors=blue]{hyperref}
\usepackage{graphicx}
\usepackage{amssymb}
\usepackage{xcolor}

\usepackage{tcolorbox}
\tcbuselibrary{theorems}

% Aggiungo pgfplots per disegnare i grafici
\usepackage{pgfplots}
\pgfplotsset{compat=1.18}

% Questo crea un nuovo tcolorbox per le definizioni
\newtcbtheorem{definition}{Definizione} % {nome box}{nome visualizzato}
{
    colback=magenta!10!white,
    colframe=magenta!80!black,
    fonttitle=\bfseries
}
{def} % Questo è un prefisso per la label

\newtcbtheorem{theorem}{Teorema} % {nome box}{nome visualizzato}
{
    colback=blue!10!white,
    colframe=blue!80!black,
    fonttitle=\bfseries
}
{thm}

% Box per le spiegazioni
\newtcbtheorem{explanation}{Spiegazione}{
    colback=green!10!white,
    colframe=green!60!black,
    fonttitle=\bfseries
}
{expl}