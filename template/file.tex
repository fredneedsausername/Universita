\documentclass{article}

\usepackage{amsmath}
\usepackage[italian]{babel}
\usepackage[a4paper,top=2cm,bottom=2cm,left=3cm,right=3cm,marginparwidth=1.75cm]{geometry}
\usepackage[colorlinks=true, allcolors=blue]{hyperref}
\usepackage{graphicx}
\usepackage{amssymb}
\usepackage{xcolor}

\title{Titolo}
\author{Federico Veronesi}

\begin{document}
\maketitle

{
  \hypersetup{linkcolor=black}
  \tableofcontents
}

\begin{abstract}
Abstract del documento.
\end{abstract}

\section{Sezioni}

Lorem ipsum dolor sit amet, consectetur adipiscing elit. Curabitur dictum augue vitae arcu finibus, vitae pharetra neque gravida. Integer posuere, risus at dapibus egestas, arcu felis dapibus arcu, quis rhoncus nulla justo quis magna. Cras id urna id risus bibendum maximus. Sed hendrerit, lacus vitae fermentum porttitor, augue nulla posuere justo, eget gravida nunc est id arcu. Phasellus vel \href{google.com}{Il sito di google} lectus id lorem pharetra volutpat.

Vivamus euismod, velit ut luctus blandit, nunc elit ultricies est, id porta quam mi sit amet eros. Suspendisse potenti. Donec eu sapien ut ipsum pharetra accumsan. Quisque consequat, magna et dictum sodales, lorem risus lacinia ante, non convallis magna lacus in felis. Maecenas a risus sit amet lorem iaculis interdum. Nam eu sem sit amet lectus rhoncus rutrum.

\section{Sezionissime}

\subsection{Sezionissimissime}

Aenean congue turpis sed tortor congue, vitae pretium nunc fermentum. Curabitur pulvinar, nibh quis mattis tristique, erat lorem aliquam massa, nec ultrices mi nibh non nibh. Aliquam erat volutpat. Pellentesque dignissim, neque in convallis pulvinar, magna magna convallis dolor, non condimentum nulla arcu a leo. Etiam id lectus ac sem tristique feugiat. Nulla facilisi.

\subsection{Sezionissimi\dots ssime}

Proin feugiat feugiat nunc, ut elementum ex gravida id. Sed et libero nec lectus efficitur gravida. Phasellus ultricies, nibh sit amet interdum congue, erat lacus feugiat justo, non convallis mauris nibh a massa. In venenatis magna quis risus hendrerit, quis placerat nunc imperdiet. Curabitur tempus, sem vitae sagittis porta, turpis mi imperdiet turpis, quis tristique dui arcu at ipsum.

\subsection{Liste}

\begin{enumerate}
\item Sono il primo! Haha
\item Oh no, sono il secondo...
\end{enumerate}
Intermissione
\begin{itemize}
\item Cavoli
\item Fagioli
\end{itemize}

\subsection{Bro famme scrive' npo de mate daje}

$X_1, X_2, \ldots, X_n$ sono delle icchese $\text{E}[X_i] = \mu$ wow! $\text{Var}[X_i] = \sigma^2 < \infty$, ma non hai visto ancora niente
\[S_n = \frac{X_1 + X_2 + \cdots + X_n}{n}
      = \frac{1}{n}\sum_{i}^{n} X_i\]
Bla bla ecco una enne $n$ un po' di testo ed ecco che $\sqrt{n}(S_n - \mu)$ super mega -> $\mathcal{N}(0, \sigma^2)$.

\end{document}
